\documentclass[12pt]{scrartcl}


\usepackage{epsfig,amssymb}

\usepackage{xcolor}
\usepackage{graphicx}
\usepackage{epstopdf}
\usepackage{multirow}
\usepackage{float}

\definecolor{darkred}{rgb}{0.5,0,0}
\definecolor{darkgreen}{rgb}{0,0.5,0}
\usepackage[pdfusetitle]{hyperref}
\hypersetup{
  letterpaper,
  colorlinks,
  linkcolor=red,
  citecolor=darkgreen,
  menucolor=darkred,
  urlcolor=blue,
  pdfpagemode=none,
}

\usepackage{fullpage}
\usepackage{tikz}
\pagestyle{empty} %
%obsolete: \usepackage{subfigure}
%use: 
\usepackage{subcaption}

\definecolor{MyDarkBlue}{rgb}{0,0.08,0.45}
\definecolor{MyDarkRed}{rgb}{0.45,0.08,0}
\definecolor{MyDarkGreen}{rgb}{0.08,0.45,0.08}

\definecolor{mintedBackground}{rgb}{0.95,0.95,0.95}
\definecolor{mintedInlineBackground}{rgb}{.90,.90,1}

\usepackage[newfloat=true]{minted}

\setminted{mathescape,
           linenos,
           autogobble,
           frame=none,
           framesep=2mm,
           framerule=0.4pt,
           %label=foo,
           xleftmargin=2em,
           xrightmargin=0em,
           %startinline=true,  %PHP only, allow it to omit the PHP Tags *** with this option, variables using dollar sign in comments are treated as latex math
           numbersep=10pt, %gap between line numbers and start of line
           style=default} %syntax highlighting style, default is "default"

\setmintedinline{bgcolor={mintedBackground}}
%doesn't work with the above workaround:
\setminted{bgcolor={mintedBackground}}
\setminted[text]{bgcolor={mintedBackground},linenos=false,autogobble,xleftmargin=1em}
%\setminted[php]{bgcolor=mintedBackgroundPHP} %startinline=True}
\SetupFloatingEnvironment{listing}{name=Code Sample}
\SetupFloatingEnvironment{listing}{listname=List of Code Samples}

\setlength{\parindent}{0pt} %
\setlength{\parskip}{.25cm}
\newcommand{\comment}[1]{}

\usepackage{amsmath}
\usepackage{algorithm2e}
\SetKwInOut{Input}{input}
\SetKwInOut{Output}{output}
%NOTE: you can embed algorithms in solutions, but they cannot be floating objects; use [H] to make them non-floats

\usepackage{lastpage}

%\usepackage{titling}
\usepackage{fancyhdr}
\renewcommand*{\titlepagestyle}{fancy}
\pagestyle{fancy}
%\fancyhf{}
%\rhead{Computer Science I}
%\lhead{Guides and tutorials}
\renewcommand{\headrulewidth}{0.0pt}
\renewcommand{\footrulewidth}{0.4pt}
\lfoot{\Title\ -- Computer Science I}
\cfoot{~}
\rfoot{\thepage\ / \pageref*{LastPage}}


\makeatletter
\title{Hack 5.0}\let\Title\@title
\subtitle{Computer Science I\\
{\small
\vskip1cm
Department of Computer Science \& Engineering \\
University of Nebraska--Lincoln}
\vskip-1cm}
%\author{Dr.\ Chris Bourke}
\date{~}
\makeatother

\begin{document}

\maketitle

\hrule

\section*{Introduction}

Hack session activities are small weekly programming assignments intended
to get you started on full programming assignments.  Collaboration is allowed
and, in fact, \emph{highly encouraged}.  You may start on the activity before
your hack session, but during the hack session you must either be actively 
working on this activity or \emph{helping others} work on the activity.
You are graded using the same rubric as assignments so documentation, style, 
design and correctness are all important.

%\subsection*{Rubric}
%\begin{table}[H]
%\begin{tabular}{ll}
%Category       & Point Value \\
%Style          & 2           \\
%Documentation  & 2           \\
%Design         & 5           \\
%Correctness    & 16          \\
%\textbf{Total} & \textbf{25}
%\end{tabular}
%\end{table}



\section*{Problem Statement}

To get some practice designing and using functions, you will 
create a small library of utility functions by implementing
the following functions with the given prototypes and specified 
functionality.
\begin{enumerate}
  \item \mintinline{c}{double degreesToRadians(double degree);} - Write a 
  	function to convert degrees to radians using the formula 
		$$\frac{d\cdot \pi}{180}$$
  \item Write a function to compute the air distance between two locations 
    identified by their latitude/longitude.  
\begin{minted}{c}
double getAirDistance(double originLatitude, 
                      double originLongitude, 
                      double destinationLatitude, 
                      double destinationLongitude);
\end{minted}  
The air distance between two latitude/longitude points can be calculated 
using the Spherical Law of Cosines:
 $$d = \arccos{(\sin(\varphi_1) \sin(\varphi_2) + \cos(\varphi_1) \cos(\varphi_2) \cos(\Delta) )} \cdot R$$
where
\begin{itemize}
  \item $\varphi_1$ is the latitude of location $A$, $\varphi_2$ is the latitude of location $B$
  \item $\Delta$ is the difference between location $B$'s longitude and location $A$'s longitude
  \item $R$ is the (average) radius of the earth, 6,371 kilometers
\end{itemize}
Note: the formula above assumes that latitude and longitude are measured 
in radians $r$, $-\pi \leq r \leq \pi$, but the function will expect 
the latitude/longitude to be in degrees.  Latitude should be in the range 
$[-90, 90]$ and longitude in the range $[-180, 180]$.  Negative values 
correspond to the southern and western hemispheres.

\newpage
  \item An object traveling at a velocity $v$ experiences time dilation
  relative to a stationary object which is quantified by the Lorentz equation:
  $$T = \frac{t}{\sqrt{(1-\frac{v^2}{c^2})}}$$
  where $t$ is the normal amount of lapsed time (stationary object) 
  and $T$ is the dilated time experienced by the traveling object.  
  For small velocities, the dilation is small, but for velocities
  approaching a \emph{percentage} (on the scale $[0, 1]$) of the 
  speed of light, $c$, the dilation becomes significant.
  
  For example, at 25\% the speed of light, a year for the object 
  traveling would correspond to 1.032796 years at the stationary object 
  (or nearly 12 extra days).   A person traveling at high velocity 
  would experience ``slowed'' time
  relative to the stationary \emph{frame}.  Implement a function to 
  compute the dilated time given the normal time $t$ (units may vary)
  and the percentage of the speed of light.
\begin{minted}{c}
double lorentzTimeDilation(double t, double percentC);
\end{minted}

\end{enumerate}  

\section*{Instructions}

\begin{itemize}
  \item You are encouraged to collaborate any number of students 
  before, during, and after your scheduled hack session.  
  \item Design at least 3 test cases for each function
  \emph{before} you begin
  designing or implementing your program.  Test cases are 
  input-output pairs that are known to be correct using means
  other than your program.
  \item Include the name(s) of everyone who worked together on
  this activity in your source file's header.
  \item Place your prototypes and documentation in a header file 
  named \mintinline{text}{utils.h} and your source in a file
  named \mintinline{text}{utils.c}.
  \item In addition, implement all of your test cases in a
  \emph{test driver} file named \mintinline{text}{utilsTester.c}
  which should output the expected output, the actual output and
  a message on whether or not the test case passed.  You must have
  at least 3 test cases for \emph{each} of your functions.
  You should \emph{not} prompt for input or use command line 
  arguments.  Your test cases should be hardcoded in your test 
  driver's \mintinline{c}{main} function.  
  \item Turn in all of your files via webhandin, making sure that 
  it runs and executes correctly in the webgrader.  Each individual 
  student will need to hand in their own copy and will receive 
  their own individual grade.
\end{itemize}
  


\end{document}
