\documentclass[12pt]{scrartcl}


\usepackage{epsfig,amssymb}

\usepackage{xcolor}
\usepackage{graphicx}
\usepackage{epstopdf}
\usepackage{multirow}
\usepackage{float}

\definecolor{darkred}{rgb}{0.5,0,0}
\definecolor{darkgreen}{rgb}{0,0.5,0}
\usepackage[pdfusetitle]{hyperref}
\hypersetup{
  letterpaper,
  colorlinks,
  linkcolor=red,
  citecolor=darkgreen,
  menucolor=darkred,
  urlcolor=blue,
  pdfpagemode=none,
}

\usepackage{fullpage}
\usepackage{tikz}
\pagestyle{empty} %
%obsolete: \usepackage{subfigure}
%use: 
\usepackage{subcaption}

\definecolor{MyDarkBlue}{rgb}{0,0.08,0.45}
\definecolor{MyDarkRed}{rgb}{0.45,0.08,0}
\definecolor{MyDarkGreen}{rgb}{0.08,0.45,0.08}

\definecolor{mintedBackground}{rgb}{0.95,0.95,0.95}
\definecolor{mintedInlineBackground}{rgb}{.90,.90,1}

\usepackage[newfloat=true]{minted}

\setminted{mathescape,
           linenos,
           autogobble,
           frame=none,
           framesep=2mm,
           framerule=0.4pt,
           %label=foo,
           xleftmargin=2em,
           xrightmargin=0em,
           %startinline=true,  %PHP only, allow it to omit the PHP Tags *** with this option, variables using dollar sign in comments are treated as latex math
           numbersep=10pt, %gap between line numbers and start of line
           style=default} %syntax highlighting style, default is "default"

\setmintedinline{bgcolor={mintedBackground}}
%doesn't work with the above workaround:
\setminted{bgcolor={mintedBackground}}
\setminted[text]{bgcolor={mintedBackground},linenos=false,autogobble,xleftmargin=1em}
%\setminted[php]{bgcolor=mintedBackgroundPHP} %startinline=True}
\SetupFloatingEnvironment{listing}{name=Code Sample}
\SetupFloatingEnvironment{listing}{listname=List of Code Samples}

\setlength{\parindent}{0pt} %
\setlength{\parskip}{.25cm}
\newcommand{\comment}[1]{}

\usepackage{amsmath}
\usepackage{algorithm2e}
\SetKwInOut{Input}{input}
\SetKwInOut{Output}{output}
%NOTE: you can embed algorithms in solutions, but they cannot be floating objects; use [H] to make them non-floats

\usepackage{lastpage}

%\usepackage{titling}
\usepackage{fancyhdr}
\renewcommand*{\titlepagestyle}{fancy}
\pagestyle{fancy}
%\fancyhf{}
%\rhead{Computer Science I}
%\lhead{Guides and tutorials}
\renewcommand{\headrulewidth}{0.0pt}
\renewcommand{\footrulewidth}{0.4pt}
\lfoot{\Title\ -- Computer Science I}
\cfoot{~}
\rfoot{\thepage\ / \pageref*{LastPage}}


\makeatletter
\title{Hack 14.0}\let\Title\@title
\subtitle{Computer Science I\\
Data Processing Part I\\
{\small
\vskip1cm
Department of Computer Science \& Engineering \\
University of Nebraska--Lincoln}
\vskip-3cm}
%\author{Dr.\ Chris Bourke}
\date{~}
\makeatother

\begin{document}

\maketitle

\hrule

\section*{Introduction}

Hack session activities are small weekly programming assignments intended
to get you started on full programming assignments.  Collaboration is allowed
and, in fact, \emph{highly encouraged}.  You may start on the activity before
your hack session, but during the hack session you must either be actively 
working on this activity or \emph{helping others} work on the activity.
You are graded using the same rubric as assignments so documentation, style, 
design and correctness are all important.

%\subsection*{Rubric}
%\begin{table}[H]
%\begin{tabular}{ll}
%Category       & Point Value \\
%Style          & 2           \\
%Documentation  & 2           \\
%Design         & 5           \\
%Correctness    & 16          \\
%\textbf{Total} & \textbf{25}
%\end{tabular}
%\end{table}



Correctness:
\begin{itemize}
  \item 4 points each test case, it is okay to be off by 1 cent
\end{itemize}


\section*{Problem Statement}


Data processing is fundamental to Computer Science and many other disciplines.
In this hack you will start a mini-project in which you will process a large 
amount of transaction data from a financial institution.  The data represents 
transactions that transfer funds from or to an account.\footnote{The data we 
use is derived from financial data generated by PaySim, a transaction 
simulator, see \url{https://www.kaggle.com/ntnu-testimon/paysim1}}


\subsection*{The Data}

Transaction data is represented as a flat file in CSV format.  The first line
indicates how many records are contained in the file.  Each subsequent line 
represents a single transaction.  A transaction includes the following pieces
of data:
\begin{itemize}
  \item A universally unique identifier (UUID), an alphanumeric designation 
    that uniquely identifies the transaction
  \item The type of transaction, which may be one of the following:
    \mintinline{text}{PAYMENT}, \mintinline{text}{TRANSFER}, \mintinline{text}{WITHDRAWAL}, 
    \mintinline{text}{DEBIT}, \mintinline{text}{DEPOSIT}
  \item The amount of the transaction
  \item The customer account number
  \item The customer account balance before the transaction
  \item The customer account balance after the transaction
  \item The transfer account number or merchant designation identifying where the funds are sent 
\end{itemize}
A small example is provided in Figure \ref{figure:dataFile}.

\begin{figure}[ht]
\begin{minted}[fontsize=\scriptsize]{text}
10
BC377639-37CC-4824-81AF-60177418B46D,PAYMENT,14535.18,C1906093041,83310.00,68774.82,M95867054
A35686EA-BFE7-455E-A027-54D73968E6D3,PAYMENT,11367.98,C371579810,77199.32,65831.34,M1932650331
7721D4C1-5A2E-4AA0-8AC3-351DFC4FE84A,DEPOSIT,159266.75,C1387043512,7777736.40,7937003.15,C553899299
F8A44740-DE85-4477-9E5E-C090AABE0BF2,WITHDRAWAL,79628.70,C739752704,203042.00,123413.30,C579345824
043FD663-236E-4C02-B042-E2642AA6471E,DEBIT,3225.27,C750937249,99375.00,96149.73,C1232504595
E0EB18EC-4EE2-4206-8834-9465992D1B28,PAYMENT,11268.73,C385148617,167267.40,155998.67,M1084205719
C306CA7E-80A0-499E-8CDE-13DFB50C6753,PAYMENT,17526.96,C1474376627,332543.08,315016.12,M1944361847
003C45B2-1356-4C7F-8D8A-EBB7BF026F16,DEPOSIT,149504.20,C386278926,56082.00,205586.20,C1113941243
B7ECC2F3-E455-4EF8-B4DF-55150A9F3B1F,PAYMENT,7676.11,C2018933315,542758.39,535082.28,M1018630839
76947698-1302-47D9-9CAF-66F21A2A6A52,PAYMENT,3240.99,C208944046,4778.00,1537.01,M962275185
\end{minted}
\caption{Example data file.}
\label{figure:dataFile}
\end{figure}

\subsection*{Modeling \& Loading the Data}

To start, you'll need to write a C program that reads and processes 
a data file containing financial transactions in the format described above.  
You should design and implement a good model for this data as you'll use 
this code as the basis for producing several reports.

In addition, you should design at least 1 non-trivial test input file that 
has at least 2 examples of each type of transaction.  You should name your
file \mintinline{text}{transactions.csv} and hand it in.

\subsection*{Reporting the Data}

To ensure that your data parsing works, you'll need to produce a report
that, for each type of transaction, produces a total number of transactions 
as well as a total of all amounts of those transactions.  For example, 
the input in Figure \ref{figure:dataFile} should produce a report that
may look like the output in Figure \ref{figure:simpleReport}.

\begin{figure}[ht]
\begin{minted}{text}
=======================================
Totals Report
=======================================
Type            Count             Total
=======================================
Payment             6  $       65615.95
Transfer            0  $           0.00
Withdraw            1  $       79628.70
Debit               1  $        3225.27
Deposit             2  $      308770.95
=======================================
Total              10  $      457240.87
\end{minted}
\caption{Simple aggregate report output example}
\label{figure:simpleReport}
\end{figure}

Your program should take the name of the input file as a command line argument
and output the report to the standard output.


\section*{Instructions}

\begin{itemize}

  \item Place all of your function definitions in a source file named 
  \mintinline{text}{transaction.c} and hand it in with your header file, 
  \mintinline{text}{transaction.h}.  Your main function should be
  placed in a file named \mintinline{text}{transactionReport.c}.  This
  function will produce the report above.

  \item Turn in your test case file, \mintinline{text}{transactions.csv}
  to the webhandin as well.

  \item You are encouraged to collaborate with any number of students 
  before, during, and after your scheduled hack session.  

  \item Include the name(s) of everyone who worked together on
  this activity in your source file's header.

  \item Turn in all of your files via webhandin, making sure that 
  it runs and executes correctly in the webgrader.  Each individual 
  student will need to hand in their own copy and will receive 
  their own individual grade.
\end{itemize}  


\end{document}
