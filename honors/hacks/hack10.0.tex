\documentclass[12pt]{scrartcl}


\usepackage{epsfig,amssymb}

\usepackage{xcolor}
\usepackage{graphicx}
\usepackage{epstopdf}
\usepackage{multirow}
\usepackage{float}

\definecolor{darkred}{rgb}{0.5,0,0}
\definecolor{darkgreen}{rgb}{0,0.5,0}
\usepackage[pdfusetitle]{hyperref}
\hypersetup{
  letterpaper,
  colorlinks,
  linkcolor=red,
  citecolor=darkgreen,
  menucolor=darkred,
  urlcolor=blue,
  pdfpagemode=none,
}

\usepackage{fullpage}
\usepackage{tikz}
\pagestyle{empty} %
%obsolete: \usepackage{subfigure}
%use: 
\usepackage{subcaption}

\definecolor{MyDarkBlue}{rgb}{0,0.08,0.45}
\definecolor{MyDarkRed}{rgb}{0.45,0.08,0}
\definecolor{MyDarkGreen}{rgb}{0.08,0.45,0.08}

\definecolor{mintedBackground}{rgb}{0.95,0.95,0.95}
\definecolor{mintedInlineBackground}{rgb}{.90,.90,1}

\usepackage[newfloat=true]{minted}

\setminted{mathescape,
           linenos,
           autogobble,
           frame=none,
           framesep=2mm,
           framerule=0.4pt,
           %label=foo,
           xleftmargin=2em,
           xrightmargin=0em,
           %startinline=true,  %PHP only, allow it to omit the PHP Tags *** with this option, variables using dollar sign in comments are treated as latex math
           numbersep=10pt, %gap between line numbers and start of line
           style=default} %syntax highlighting style, default is "default"

\setmintedinline{bgcolor={mintedBackground}}
%doesn't work with the above workaround:
\setminted{bgcolor={mintedBackground}}
\setminted[text]{bgcolor={mintedBackground},linenos=false,autogobble,xleftmargin=1em}
%\setminted[php]{bgcolor=mintedBackgroundPHP} %startinline=True}
\SetupFloatingEnvironment{listing}{name=Code Sample}
\SetupFloatingEnvironment{listing}{listname=List of Code Samples}

\setlength{\parindent}{0pt} %
\setlength{\parskip}{.25cm}
\newcommand{\comment}[1]{}

\usepackage{amsmath}
\usepackage{algorithm2e}
\SetKwInOut{Input}{input}
\SetKwInOut{Output}{output}
%NOTE: you can embed algorithms in solutions, but they cannot be floating objects; use [H] to make them non-floats

\usepackage{lastpage}

%\usepackage{titling}
\usepackage{fancyhdr}
\renewcommand*{\titlepagestyle}{fancy}
\pagestyle{fancy}
%\fancyhf{}
%\rhead{Computer Science I}
%\lhead{Guides and tutorials}
\renewcommand{\headrulewidth}{0.0pt}
\renewcommand{\footrulewidth}{0.4pt}
\lfoot{\Title\ -- Computer Science I}
\cfoot{~}
\rfoot{\thepage\ / \pageref*{LastPage}}


\makeatletter
\title{Hack 10.0}\let\Title\@title
\subtitle{Computer Science I - Java\\
File I/O\\
{\small
\vskip1cm
Department of Computer Science \& Engineering \\
University of Nebraska--Lincoln}
\vskip-2cm}
%\author{Dr.\ Chris Bourke}
\date{~}
\makeatother

\begin{document}

\maketitle

\hrule

\section*{Introduction}

Hack session activities are small weekly programming assignments intended
to get you started on full programming assignments.  Collaboration is allowed
and, in fact, \emph{highly encouraged}.  You may start on the activity before
your hack session, but during the hack session you must either be actively 
working on this activity or \emph{helping others} work on the activity.
You are graded using the same rubric as assignments so documentation, style, 
design and correctness are all important.

%\subsection*{Rubric}
%\begin{table}[H]
%\begin{tabular}{ll}
%Category       & Point Value \\
%Style          & 2           \\
%Documentation  & 2           \\
%Design         & 5           \\
%Correctness    & 16          \\
%\textbf{Total} & \textbf{25}
%\end{tabular}
%\end{table}



\section*{Exercises}

To get more practice working with files, you will write several 
methods that involve operations on files.  In particular, implement
the following functions.

\begin{enumerate}

  \item Write a method that, given a file path/name as a string opens
  the file and returns its entire contents as a single string.  Any endline
  characters should \emph{be preserved}.
  
  \mintinline{java}{public static String getFileContents(String filePath)}

  \item Write a method that, given a file path/name as a string opens
  the file and returns the contents of the file as a list of strings.
  Each element in the list should correspond to a line in the file.
  Any end line character should be \emph{chomped out} and not included.

  \mintinline{java}{public static List<String> getFileLines(String filePath)}

\end{enumerate}

\subsection*{Protein Translation}

DNA is a molecule that encodes genetic information.  A DNA sequence is 
a string of nucleotides represented as letters A, T, C, and G (representing
the nucleobases adenine, thymine, cytosine, and guanine respectively).  
Protein sequencing in an organism consists of a two step process.  First 
the DNA is translated into RNA by replacing each thymine nucleotide with 
uracil (U).  Then, the RNA sequence is translated into a protein (a sequence
of amino acids) according to the following rules.

The RNA sequence is processed 3 bases at a time called a \emph{codon}.  
Each codon is translated into a single amino acid according to known 
encoding rules.  There are 20 such amino acids, each represented by a 
single letter in 
 $$(A,C,D,E,F,G,H,I,K,L,M,N,P,Q,R,S,T,V,W,Y)$$
Because there are $4^3 = 64$ possible codons but only 20 amino acids,
some codons translate to the same amino acid.

The rules for translating trigrams are complex, but we've simplified
the process by providing starter code that includes a \mintinline{java}{Map}
which is an extremely useful data structure that allows you to map
keys to values.  In this case, the \mintinline{java}{Map} maps
RNA codons (as strings) to proteins (as a single character string).  
If you provide it an invalid sequence, it will return \mintinline{java}{null}.

In addition, the trigrams UAA, UAG, and UGA are special markers that 
indicate a (premature) end to the protein sequencing (there may be 
additional nucleotides left in the RNA sequence, but they are ignored 
and the translation ends).  The \mintinline{java}{Map} we've provided will
return a lower-case \mintinline{c}{x} character for any of these three 
trigrams.

As an example, suppose we start with the DNA sequence $AAATTCCGCGTACCC$; 
it would be encoded into RNA as $AAAUUCCGCGUACCC$; and into an amino 
acid sequence $KFRVP$.

You will write a program that takes two command line arguments.  The
first is an input file containing a DNA sequence and the second is the
name of the output file in which you'll place the translated protein
sequence.  

The input file \emph{may} contain irrelevant whitespace characters to
avoid very long lines.  You will need to \emph{ignore} any whitespace
characters when you process the data.  

\section*{Instructions}

\begin{itemize}

  \item For the exercises, place all your methods into a source file 
  named \mintinline{text}{FileUtils.java} with proper documentation.
  In addition, you'll want
  to create a main test driver program that demonstrates at least 3 cases 
  per method to verify their output.  You need not hand it in, however.
  
  \item Code for the RNA-to-protein \mintinline{java}{Map} as well as 
  a demonstration on how to use it has been provided in the
  \mintinline{text}{ProteinTranslator.java} source file.  You should
  add your translation code using this class.

  \item \textbf{Hint}: Code reuse is a Very Good Thing.  Your protein program
  can use your file utility methods, but also: feel free to include
  any additional functions you may have written before in the 
  \mintinline{text}{FileUtils.java} file and use them in your protein translator program.

  \item You are encouraged to collaborate any number of students 
  before, during, and after your scheduled hack session.  

  \item You may (in fact are encouraged) to define any additional
  ``helper'' methods that may help you.

  \item Include the name(s) of everyone who worked together on
  this activity in your source file's header.

  \item Turn in all of your files via webhandin, making sure that 
  it runs and executes correctly in the webgrader.  Each individual 
  student will need to hand in their own copy and will receive 
  their own individual grade.
\end{itemize}  


\end{document}
